\documentclass[fleqn,usenatbib]{mnras}
\usepackage[T1]{fontenc}
\usepackage{ae,aecompl}
\usepackage{graphicx}   % Including figure files
\usepackage{amsmath}    % Advanced maths commands
\usepackage{amssymb}    % Extra maths symbols
\usepackage{siunitx}    % for numbers with units
\DeclareSIUnit{\degree}{deg} % we want degrees written with 'deg' and
                             % not with a superscript circle.
\DeclareSIUnit{\arcmin}{arcmin} % we want degrees written  'arcmin' and
                             % not as a prime.
\DeclareSIUnit{\arcsec}{arcsec} % we want arcsec written  'arcsec' and
                             % not as a double prime.
\usepackage{color,verbatim,url}
\usepackage{tabularx}
\usepackage[table,usenames,dvipsnames]{xcolor}
\usepackage{booktabs}
\usepackage{pdflscape}
\usepackage{lastpage}
\usepackage{listings}
\usepackage{xcolor}
\lstset{escapeinside={<@}{@>}}

\usepackage{minitoc}

\definecolor{pink}{rgb}{0.858, 0.188, 0.478}
\definecolor{purple}{RGB}{76, 0,153}
\def\ie{\,{\rm i.e.}\,}
\def\eg{\,{\rm e.g.}\,}
\onecolumn

\newcommand{\red}[1]{{\color{red}{#1}}}
\newcommand{\green}[1]{{\color{green}{#1}}}
\newcommand{\blue}[1]{{\color{blue}{#1}}}
\newcommand{\x}{\vec{x}}
\renewcommand{\d}[0]{{\rm d}}

\newcommand{\be}{\begin{equation}}  \newcommand{\ee}{\end{equation}}
\newcommand{\mb}[1]{\mbox{ #1 }}  
\newcommand{\ba}{\begin{eqnarray}}\newcommand{\ea}{\end{eqnarray}}
\newcommand{\bm}[1]{\mbox{\boldmath{$#1$}}}   %this is bold italic for MNRAS
\newcommand{\mat}[1]{\mathbfss{#1}}

\title[CosmoPipe Manual]{CosmoPipe Manual} 

\date{Compiled: @DATE@}

\begin{document}
\setlength{\voffset}{-12mm}

\doparttoc % Tell to minitoc to generate a toc for the parts
\faketableofcontents % Run a fake tableofcontents command for the partocs
\noptcrule  %No horizontal rules in TOC

\maketitle

\section{Installation}
CosmoPipe can be installed on linux distributions using the command: 

\section{Execution} 
After installation, you will receive a prompt that contains all pertinent information for running CosmoPipe. 
It looks similar to the below: 

\begin{verbatim}
CosmoPipe has been installed at the below path:
/path/to/your/running/directory

In that directory, you will find: 
  - configure.sh (The pipeline configuration script)
  - variables.sh (The file contains compile-time variables)
    -> These variables cannot be edited after compilation.
  - defaults.sh (The file contains run-time variables)
    -> These variables can be assigned/modified as the pipeline runs, and so this file just
       contains the global default values that they are assigned during compilation. 
    -> Only run-time variables needed by your pipeline are important, and so the compilation
       will select the needed run-time variables and put them into a different bespoke file.
       So you can probably ignore this file for now.
  - pipeline.ini (The pipeline definition script)

To use CosmoPipe: 
  1) Go to the directory containing CosmoPipe (listed above)
  2) Check the variables.sh file has all the variables correctly defined
    -> Of particular importance is the PIPELINE variable, which tells
       CosmoPipe which pipeline in pipeline.ini to construct!
  3) Check the pipeline.ini file has your desired pipeline, and that the pipeline is correct
  4) Run the configuration: conda run -n cosmopipe bash configure.sh 
  5) Follow the configuration instructions! 
\end{verbatim}

\section{Pipelines} 
A pipeline is a list of processing functions. An example of a simple pipeline is below in plain text:
\begin{enumerate}
  \item[\textbf{MyPipeline:}]
  \item[Step 1:] Load in an LDAC-format catalogue 
  \item[Step 2:] Split the catalogue into line-of-sight (`tomographic') bins 
  \item[Step 3:] Remove sources that are fainter than some magnitude limit 
  \item[Step 4:] Compute and subtract the mean ellipticity (`c-term') from all sources in each bin 
\end{enumerate}
Such a pipeline can be constructed in CosmoPipe using similar plain-text, by
calling relevant processing functions that achieve each of these goals. A CosmoPipe 
pipeline for `MyPipeline', as it would appear in the \texttt{pipeline.ini} file, 
is given below:
\begin{verbatim}
MyPipeline: 
  add_head 
  tomography
  ldacfilter
  correct_cterm
\end{verbatim}
Of course, there are a number of parameters that must be set for this pipeline to work. 

\begin{itemize} 
  \item[Step 1:] What catalogue should be added? 
  \item[Step 2:] How should the tomographic bins be defined? 
  \item[Step 3:] How should the magnitude limit be defined? 
  \item[Step 4:] What are the shape-measurement variable names? 
\end{itemize}
This is where CosmoPipe helps the user and lowers the barrier to entry for complex analyses. 
If we ignore the above questions and just blindly compile the above pipeline (by running 
\texttt{bash compile.sh MyPipeline}), we see the below output: 

\begin{verbatim} 
======================================================
==         Cosmology Pipeline Configuration         ==
======================================================
> Copying Provided Data Products to Storage Path - Done!
> Copying scripts & configs to Run directory - Done!
> Modify Runtime Scripts for 325 files (166+27+132) - Done!
> Constructing Pipeline MyPipeline  {
  WARNINGS:
  The pipeline used the following undeclared runtime variables:
  E1NAME E2NAME INPUTS NBOOT TOMOLIMS TOMOVAR WEIGHTNAME
  Of these variables, the following have no default value assigned:
  INPUTS FILTERCOND 
  You need to update those variables in the MyPipeline_defaults.sh file!
} - Done!
> Finished! To run the Cosmology pipeline, check cosmosis is configured (source cosmosis-configure) and run:
  bash MyPipeline_pipeline.sh  (from within the 'cosmopipe' conda env), or:
  conda run -n  --no-capture-output bash MyPipeline_pipeline.sh  (from anywhere).
\end{verbatim}
The pipeline has notified the user that there are 9 variables which are needed to sucessfully execute this pipeline. 
However 7 of them already have default values assigned within the master defaults.sh file which ships with CosmoPipe. 
These may or may not be desireable for each users needs, but nonetheless the defaults exist and are placed into the 
bespoke \texttt{MyPipeline\_defaults.sh} file, for use by CosmoPipe. The variables "INPUTS" and "FILTERCOND", however, 
have no default, and so \textbf{must} be set to a desired value by the user in the \texttt{MyPipeline\_defaults.sh} file 
before running the pipeline. The  \texttt{MyPipeline\_defaults.sh} file looks as below after compilation: 

\begin{verbatim}[MyPipeline_defaults.sh]
#Path of file(s) to add to the datahead
INPUTS=@BV:INPUTS@
#Limits of the tomographic bins
TOMOLIMS='0.1 0.3 0.5 0.7 0.9 1.2'
#Variable used to define tomographic bins
TOMOVAR=Z_B
#Filtering condition
FILTERCOND=@BV:FILTERCOND@
#Shape measurement variables: e1
E1NAME=autocal_e1
#Shape measurement variables: e2
E2NAME=autocal_e2
#Number of bootstrap realisations
NBOOT=300
#Name of the lensing weight variable
WEIGHTNAME=weight
\end{verbatim}

We can set the values of \texttt{INPUTS} and \texttt{FILTERCOND} to useful values in this file and then run the pipeline. 
Alternatively, though, one can set the values of these parameters in the pipeline itself, which will then avoid the warning 
about ``no default values'' (as defaults for these will no longer be needed, because we provide the values explicitly!). To 
do this we use the `$+$' variable assignment syntax in our pipeline: 
\begin{verbatim}
MyPipeline: 
  +INPUTS=/path/to/my/catalogue.cat
  add_head 
  tomography
  +FILTERCOND="r_mag <= 24"
  ldacfilter
  correct_cterm
\end{verbatim}
where \texttt{r\_mag} is a column in \texttt{catalogue.cat}. When we compile this pipeline we instead see the below prompt: 
\begin{verbatim} 
======================================================
==         Cosmology Pipeline Configuration         ==
======================================================
> Copying Provided Data Products to Storage Path - Done!
> Copying scripts & configs to Run directory - Done!
> Modify Runtime Scripts for 325 files (166+27+132) - Done!
> Constructing Pipeline MyPipeline  {
  WARNINGS:
  The pipeline used the following undeclared runtime variables:
  E1NAME E2NAME INPUTS NBOOT TOMOLIMS TOMOVAR WEIGHTNAME
  !BUT! - all of them were assigned defaults in the MyPipeline_defaults.sh file!
  So there is no action is required!
} - Done!
> Finished! To run the Cosmology pipeline, check cosmosis is configured (source cosmosis-configure) and run:
  bash MyPipeline_pipeline.sh  (from within the 'cosmopipe' conda env), or:
  conda run -n  --no-capture-output bash MyPipeline_pipeline.sh  (from anywhere).
\end{verbatim}
We can then execute this pipeline in the knowledge that all data products and variables that are 
needed in the execution of the pipeline are present. 

\subsection{Subroutines} 
For more complex pipelines, the pipeline length can become excessive and/or may need to run collections of 
processing 


\section{Adding Functions} 


\section{Available Functions} 
The remainder of this document contains a list of all functions available for execution within CosmoPipe. 
This list is constantly expanding. 

\input{sections}



\end{document}
